\documentclass[8pt,aspectratio=169]{beamer}
\usetheme{Madrid}
\usepackage{graphicx}
\usepackage{booktabs}
\usepackage{adjustbox}
\usepackage{multicol}
\usepackage{amsmath}

% Color definitions
\definecolor{mlblue}{RGB}{0,102,204}
\definecolor{mlpurple}{RGB}{51,51,178}
\definecolor{mllavender}{RGB}{173,173,224}
\definecolor{mllavender2}{RGB}{193,193,232}
\definecolor{mllavender3}{RGB}{204,204,235}
\definecolor{mllavender4}{RGB}{214,214,239}
\definecolor{mlorange}{RGB}{255, 127, 14}
\definecolor{mlgreen}{RGB}{44, 160, 44}
\definecolor{mlred}{RGB}{214, 39, 40}
\definecolor{mlgray}{RGB}{127, 127, 127}

% Additional colors for template compatibility
\definecolor{lightgray}{RGB}{240, 240, 240}
\definecolor{midgray}{RGB}{180, 180, 180}

% Apply custom colors to Madrid theme
\setbeamercolor{palette primary}{bg=mllavender3,fg=mlpurple}
\setbeamercolor{palette secondary}{bg=mllavender2,fg=mlpurple}
\setbeamercolor{palette tertiary}{bg=mllavender,fg=white}
\setbeamercolor{palette quaternary}{bg=mlpurple,fg=white}

\setbeamercolor{structure}{fg=mlpurple}
\setbeamercolor{section in toc}{fg=mlpurple}
\setbeamercolor{subsection in toc}{fg=mlblue}
\setbeamercolor{title}{fg=mlpurple}
\setbeamercolor{frametitle}{fg=mlpurple,bg=mllavender3}
\setbeamercolor{block title}{bg=mllavender2,fg=mlpurple}
\setbeamercolor{block body}{bg=mllavender4,fg=black}

% Remove navigation symbols
\setbeamertemplate{navigation symbols}{}

% Clean itemize/enumerate
\setbeamertemplate{itemize items}[circle]
\setbeamertemplate{enumerate items}[default]

% Reduce margins for more content space
\setbeamersize{text margin left=5mm,text margin right=5mm}

% Command for bottom annotation (Madrid-style)
\newcommand{\bottomnote}[1]{%
\vfill
\vspace{-2mm}
\textcolor{mllavender2}{\rule{\textwidth}{0.4pt}}
\vspace{1mm}
\footnotesize
\textbf{#1}
}

\title{Beamer Template Collection}
\subtitle{22 Professional Slide Layouts with Madrid Theme}
\author{Template System}
\institute{Academic \& Professional Presentations}
\date{\today}

\begin{document}

% ==================== LAYOUT 1: PLAIN TITLE ====================
\begin{frame}[plain]
\vspace{2cm}
\begin{center}
{\Huge Main Title}\\[0.5cm]
{\Large Subtitle or Description}\\[2cm]
{\normalsize Additional Information}
\end{center}
\end{frame}

% ==================== LAYOUT 2: STANDARD TITLE ====================
\begin{frame}[plain]
\titlepage
\end{frame}

% ==================== TABLE OF CONTENTS ====================
\begin{frame}[t]{Template Overview}
\tableofcontents
\vfill
\footnotesize
\textbf{Template Structure:} This template provides 22 pre-designed slide layouts organized into sections. Each layout serves a specific purpose for academic and professional presentations. Sections include content layouts, visual formats, comparisons, specialized formats, and data visualization options.
\end{frame}

% ==================== SECTION: CONTENT LAYOUTS ====================
\section{Content Layouts}

\begin{frame}[t]
\vfill
\centering
\begin{beamercolorbox}[sep=8pt,center]{title}
\usebeamerfont{title}\Large Content Layouts\par
\end{beamercolorbox}
\vfill
\end{frame}

% ==================== LAYOUT 3: TWO COLUMNS TEXT ====================
\begin{frame}[t]{Two Column Layout - Text}
\begin{columns}[T]
\column{0.48\textwidth}
\textbf{Left Column Header}

Main content for the left side. This is where your primary information goes.

Key points:
\begin{itemize}
\item First point
\item Second point
\item Third point with more text
\item Fourth point
\end{itemize}

Additional paragraph text can go here to provide more context or explanation.

\column{0.48\textwidth}
\textbf{Right Column Header}

Supporting content or contrasting information for the right side.

Related items:
\begin{itemize}
\item Supporting point one
\item Supporting point two
\item Supporting point three
\end{itemize}

More descriptive text that complements the left column content.
\end{columns}

\bottomnote{Bottom annotation: Additional notes, references, or key takeaways}
\end{frame}

% ==================== LAYOUT 4: TWO COLUMNS WITH MATH ====================
\begin{frame}[t]{Two Column Layout - Mathematics}
\begin{columns}[T]
\column{0.48\textwidth}
\textbf{Definition}

A mathematical concept defined:
$$f(x) = ax^2 + bx + c$$

Properties:
\begin{itemize}
\item Property one: $a \neq 0$
\item Property two: Vertex at $x = -\frac{b}{2a}$
\item Property three: Discriminant $\Delta = b^2 - 4ac$
\end{itemize}

\column{0.48\textwidth}
\textbf{Example}

Specific instance:
$$f(x) = 2x^2 + 3x + 1$$

Calculation:
\begin{align*}
f'(x) &= 4x + 3 \\
f'(0) &= 3 \\
f''(x) &= 4
\end{align*}

Result: Minimum at $x = -\frac{3}{4}$
\end{columns}

\bottomnote{Mathematical concepts are best understood through both theory and examples}
\end{frame}

% ==================== LAYOUT 5: COLUMNS WITH LIST VARIATIONS ====================
\begin{frame}[t]{List Variations}
\begin{columns}[T]
\column{0.48\textwidth}
\textbf{Enumerated List}
\begin{enumerate}
\item First step in process
\item Second step with details
\item Third step
\begin{itemize}
\item Sub-point A
\item Sub-point B
\end{itemize}
\item Final step
\end{enumerate}

\vspace{0.5em}
\textbf{Bullet Points}
\begin{itemize}
\item Main concept
\item Supporting idea
\item Additional thought
\end{itemize}

\column{0.48\textwidth}
\textbf{Mixed Content}

Paragraph text introducing a concept.

Key formulas:
\begin{itemize}
\item Linear: $y = mx + b$
\item Quadratic: $y = ax^2 + bx + c$
\item Exponential: $y = ae^{bx}$
\end{itemize}

Concluding remarks about the formulas and their applications in real-world scenarios.
\end{columns}
\end{frame}

% ==================== LAYOUT 6: THREE-WAY SPLIT ====================
\begin{frame}[t]{Three Column Layout}
\begin{columns}[T]
\column{0.31\textwidth}
\textbf{Category A}

Content for first category:
\begin{itemize}
\item Item 1
\item Item 2
\item Item 3
\end{itemize}

Additional notes about this category.

\column{0.31\textwidth}
\textbf{Category B}

Content for second category:
\begin{itemize}
\item Item 1
\item Item 2
\item Item 3
\end{itemize}

Additional notes about this category.

\column{0.31\textwidth}
\textbf{Category C}

Content for third category:
\begin{itemize}
\item Item 1
\item Item 2
\item Item 3
\end{itemize}

Additional notes about this category.
\end{columns}

\bottomnote{Three columns work well for comparisons or related concepts}
\end{frame}

% ==================== SECTION: VISUAL LAYOUTS ====================
\section{Visual Layouts}

\begin{frame}[t]
\vfill
\centering
\begin{beamercolorbox}[sep=8pt,center]{title}
\usebeamerfont{title}\Large Visual Layouts\par
\end{beamercolorbox}
\vfill
\end{frame}

% ==================== LAYOUT 7: FULL WIDTH WITH IMAGE SPACE ====================
\begin{frame}[t]{Full Width Content with Image}
\textbf{Main Topic Introduction}

This layout provides space for a full-width explanation followed by an image or chart.

Key concepts to understand:
\begin{itemize}
\item Concept one with brief explanation
\item Concept two with additional details
\item Concept three relating to the visual below
\end{itemize}

\vspace{0.5em}
\begin{center}
% Space for image/chart
\framebox[0.9\textwidth][c]{
\vspace{3cm}
\textcolor{midgray}{[Image/Chart Placeholder]}
\vspace{3cm}
}
\end{center}

\bottomnote{Visuals should complement and enhance the textual content}
\end{frame}

% ==================== LAYOUT 8: COLUMNS WITH IMAGE ====================
\begin{frame}[t]{Mixed Media Layout}
\begin{columns}[T]
\column{0.48\textwidth}
\textbf{Text Content}

Explanation of concept with supporting details.

Important points:
\begin{itemize}
\item First observation
\item Second observation
\item Third observation
\item Conclusion
\end{itemize}

Formula if needed:
$$E = mc^2$$

\column{0.48\textwidth}
\begin{center}
% Space for image
\framebox[0.9\columnwidth][c]{
\vspace{4cm}
\textcolor{midgray}{[Visual Element]}
\vspace{4cm}
}
\end{center}
\end{columns}

\bottomnote{Combine text and visuals for maximum impact}
\end{frame}

% ==================== SECTION: COMPARISONS AND ANALYSIS ====================
\section{Comparisons and Analysis}

\begin{frame}[t]
\vfill
\centering
\begin{beamercolorbox}[sep=8pt,center]{title}
\usebeamerfont{title}\Large Comparisons and Analysis\par
\end{beamercolorbox}
\vfill
\end{frame}

% ==================== LAYOUT 9: DEFINITION-EXAMPLE PATTERN ====================
\begin{frame}[t]{Definition and Examples}
\begin{columns}[T]
\column{0.48\textwidth}
\textbf{Definition}

Formal statement of concept or theorem.

\vspace{0.5em}
\textbf{Properties}
\begin{itemize}
\item Property 1
\item Property 2
\item Property 3
\end{itemize}

\vspace{0.5em}
\textbf{Conditions}
\begin{itemize}
\item Must satisfy A
\item Must satisfy B
\end{itemize}

\column{0.48\textwidth}
\textbf{Example 1}

Concrete instance demonstrating the concept.

Details:
\begin{itemize}
\item Specific value: 42
\item Result: \textcolor{mlgreen}{Valid}
\end{itemize}

\vspace{0.5em}
\textbf{Example 2}

Another instance showing different aspect.

Details:
\begin{itemize}
\item Specific value: -5
\item Result: \textcolor{mlorange}{Invalid}
\end{itemize}
\end{columns}

\bottomnote{Definitions paired with examples aid understanding}
\end{frame}

% ==================== LAYOUT 10: COMPARISON TABLE ====================
\begin{frame}[t]{Comparison Layout}
\begin{columns}[T]
\column{0.48\textwidth}
\textbf{Method A}
\begin{itemize}
\item Advantage 1
\item Advantage 2
\item Advantage 3
\end{itemize}

\textbf{Disadvantages}
\begin{itemize}
\item Limitation 1
\item Limitation 2
\end{itemize}

\textbf{Best for:} Scenario type X

\column{0.48\textwidth}
\textbf{Method B}
\begin{itemize}
\item Advantage 1
\item Advantage 2
\item Advantage 3
\end{itemize}

\textbf{Disadvantages}
\begin{itemize}
\item Limitation 1
\item Limitation 2
\end{itemize}

\textbf{Best for:} Scenario type Y
\end{columns}

\bottomnote{Direct comparisons help in decision making}
\end{frame}

% ==================== LAYOUT 11: PROGRESSIVE REVEAL ====================
\begin{frame}[t]{Step-by-Step Process}
\begin{columns}[T]
\column{0.48\textwidth}
\textbf{Initial State}

Description of starting point:
\begin{itemize}
\item Given: Input data
\item Goal: Desired output
\item Constraint: Time limit
\end{itemize}

\vspace{0.5em}
\textbf{Step 1: Preparation}

Actions taken in first step.

\vspace{0.5em}
\textbf{Step 2: Execution}

Main processing occurs here.

\column{0.48\textwidth}
\textbf{Step 3: Refinement}

Optimization and adjustments.

\vspace{0.5em}
\textbf{Step 4: Validation}

Check results against criteria.

\vspace{0.5em}
\textbf{Final State}

Description of outcome:
\begin{itemize}
\item Result: \textcolor{mlgreen}{Success}
\item Time: 2.3 seconds
\item Accuracy: 99.5\%
\end{itemize}
\end{columns}

\bottomnote{Step-by-step breakdowns clarify complex processes}
\end{frame}

% ==================== LAYOUT 12: FORMULA COLLECTION ====================
\begin{frame}[t]{Formula Reference}
\begin{columns}[T]
\column{0.31\textwidth}
\textbf{Category 1}

Basic formulas:
$$a + b = c$$
$$x^2 + y^2 = r^2$$
$$F = ma$$

\column{0.31\textwidth}
\textbf{Category 2}

Intermediate formulas:
$$\int_a^b f(x)\,dx$$
$$\sum_{i=1}^n i = \frac{n(n+1)}{2}$$
$$e^{i\pi} + 1 = 0$$

\column{0.31\textwidth}
\textbf{Category 3}

Advanced formulas:
$$\nabla \times \vec{F} = 0$$
$$\frac{\partial u}{\partial t} = k\nabla^2 u$$
$$E = \hbar\omega$$
\end{columns}

\bottomnote{Quick reference formulas organized by category}
\end{frame}

% ==================== LAYOUT 13: SUMMARY STYLE ====================
\begin{frame}[t]{Summary Layout}
\begin{columns}[T]
\column{0.48\textwidth}
\textbf{Key Concepts}
\begin{itemize}
\item Main idea 1
\item Main idea 2
\end{itemize}

\vspace{0.5em}
\textbf{Methods Covered}
\begin{itemize}
\item Technique A
\item Technique B
\end{itemize}

\column{0.48\textwidth}
\textbf{Applications}
\begin{itemize}
\item Real-world use 1
\item Real-world use 2
\end{itemize}

\vspace{0.5em}
\textbf{Next Steps}
\begin{itemize}
\item Further reading
\item Advanced topics
\end{itemize}
\end{columns}

\vspace{1em}
% Visual summary placeholder - increased space
\begin{center}
\framebox[0.95\textwidth][c]{
\vspace{4.5cm}
\textcolor{midgray}{[Summary Dashboard/Visual]}
\vspace{4.5cm}
}
\end{center}

\bottomnote{Summaries consolidate learning and provide direction}
\end{frame}

% ==================== SECTION: SPECIALIZED FORMATS ====================
\section{Specialized Formats}

\begin{frame}[t]
\vfill
\centering
\begin{beamercolorbox}[sep=8pt,center]{title}
\usebeamerfont{title}\Large Specialized Formats\par
\end{beamercolorbox}
\vfill
\end{frame}

% ==================== LAYOUT 14: Q&A STYLE ====================
\begin{frame}[t]{Question and Answer Format}
\begin{columns}[T]
\column{0.48\textwidth}
\textbf{Common Questions}

\textit{Q1: What is the main purpose?}

Answer explaining the primary goal and its importance.

\vspace{0.5em}
\textit{Q2: How does it work?}

Brief explanation of the mechanism or process.

\column{0.48\textwidth}
\textit{Q3: When should it be used?}

Scenarios and conditions for application.

\vspace{0.5em}
\textit{Q4: What are the limitations?}

Known constraints and boundaries.
\end{columns}

\vspace{0.5em}
% FAQ visual placeholder
\begin{center}
\framebox[0.5\textwidth][c]{
\vspace{1.5cm}
\textcolor{midgray}{[FAQ Diagram/Icon]}
\vspace{1.5cm}
}
\end{center}

\bottomnote{Anticipating questions improves comprehension}
\end{frame}

% ==================== LAYOUT 15: CLOSING SLIDE ====================
\begin{frame}[plain]
\vspace{3cm}
\begin{center}
{\Large Thank you}\\[2cm]
{\normalsize Questions?}\\[1cm]
{\small contact@example.com}
\end{center}
\end{frame}

% ==================== LAYOUT 16: OVERVIEW WITH SECTIONS ====================
\begin{frame}[t]{Course Overview}
\begin{columns}[T]
\column{0.48\textwidth}
\textbf{Part 1: Foundations}
\begin{itemize}
\item Topic 1.1
\item Topic 1.2
\item Topic 1.3
\item Topic 1.4
\end{itemize}

\vspace{0.5em}
\textbf{Part 2: Intermediate}
\begin{itemize}
\item Topic 2.1
\item Topic 2.2
\item Topic 2.3
\end{itemize}

\column{0.48\textwidth}
\textbf{Part 3: Advanced}
\begin{itemize}
\item Topic 3.1
\item Topic 3.2
\item Topic 3.3
\end{itemize}

\vspace{0.5em}
\textbf{Part 4: Applications}
\begin{itemize}
\item Application A
\item Application B
\item Application C
\item Case Studies
\end{itemize}
\end{columns}

\vspace{0.5em}
% Course structure/roadmap visual
\begin{center}
\framebox[0.8\textwidth][c]{
\vspace{1.5cm}
\textcolor{midgray}{[Course Roadmap/Flow Diagram]}
\vspace{1.5cm}
}
\end{center}

\bottomnote{Structured overview helps learners navigate content}
\end{frame}

% ==================== LAYOUT 17: CODE AND OUTPUT ====================
\begin{frame}[t]{Code Example Layout}
\begin{columns}[T]
\column{0.48\textwidth}
\textbf{Input Code}

\texttt{def function(x):}\\
\texttt{~~~~if x > 0:}\\
\texttt{~~~~~~~~return x * 2}\\
\texttt{~~~~else:}\\
\texttt{~~~~~~~~return -x}\\
\texttt{}\\
\texttt{result = function(5)}\\
\texttt{print(result)}

\vspace{0.5em}
\textbf{Explanation}

This function doubles positive numbers and negates negative numbers.

\column{0.48\textwidth}
\textbf{Output}

\texttt{10}

\vspace{0.5em}
\textbf{Trace Through}
\begin{enumerate}
\item Input: $x = 5$
\item Check: $5 > 0$ (True)
\item Execute: $5 \times 2 = 10$
\item Return: $10$
\end{enumerate}

\textbf{Other Examples}
\begin{itemize}
\item $f(3) = 6$
\item $f(-4) = 4$
\item $f(0) = 0$
\end{itemize}
\end{columns}

\bottomnote{Code examples benefit from step-by-step explanation}
\end{frame}

% ==================== LAYOUT 18: PROS AND CONS ====================
\begin{frame}[t]{Advantages and Disadvantages}
\begin{columns}[T]
\column{0.48\textwidth}
\textbf{Advantages}
\begin{itemize}
\item[\textcolor{mlgreen}{+}] Benefit one with explanation
\item[\textcolor{mlgreen}{+}] Benefit two
\item[\textcolor{mlgreen}{+}] Benefit three
\item[\textcolor{mlgreen}{+}] Benefit four with additional context
\item[\textcolor{mlgreen}{+}] Benefit five
\end{itemize}

\column{0.48\textwidth}
\textbf{Disadvantages}
\begin{itemize}
\item[\textcolor{mlorange}{-}] Drawback one
\item[\textcolor{mlorange}{-}] Drawback two with details
\item[\textcolor{mlorange}{-}] Drawback three
\item[\textcolor{mlorange}{-}] Drawback four
\end{itemize}

\vspace{0.5em}
\textbf{Verdict}

Best suited for situations where benefits outweigh drawbacks.
\end{columns}

\bottomnote{Balanced analysis helps informed decision-making}
\end{frame}

% ==================== LAYOUT 19: TIMELINE ====================
\begin{frame}[t]{Timeline Layout}
\begin{columns}[T]
\column{0.48\textwidth}
\textbf{Phase 1: Initial Development}
\begin{itemize}
\item Week 1-2: Planning
\item Week 3-4: Design
\item Week 5-6: Prototype
\end{itemize}

\vspace{0.5em}
\textbf{Phase 2: Implementation}
\begin{itemize}
\item Week 7-10: Core features
\item Week 11-12: Testing
\item Week 13-14: Refinement
\end{itemize}

\column{0.48\textwidth}
\textbf{Phase 3: Deployment}
\begin{itemize}
\item Week 15: Beta release
\item Week 16-17: Feedback
\item Week 18: Final release
\end{itemize}

\vspace{0.5em}
\textbf{Phase 4: Maintenance}
\begin{itemize}
\item Ongoing: Updates
\item Monthly: Reviews
\item Quarterly: Major updates
\end{itemize}
\end{columns}

\vspace{0.5em}
% Timeline/Gantt chart placeholder
\begin{center}
\framebox[0.95\textwidth][c]{
\vspace{2cm}
\textcolor{midgray}{[Timeline/Gantt Chart Placeholder]}
\vspace{2cm}
}
\end{center}

\bottomnote{Clear timelines set expectations and track progress}
\end{frame}

% ==================== LAYOUT 20: REFERENCES ====================
\begin{frame}[t]{References and Resources}
\begin{columns}[T]
\column{0.48\textwidth}
\textbf{Primary Sources}
\begin{itemize}
\item Author (2024): \textit{Main Title}
\item Researcher (2023): \textit{Key Paper}
\item Expert (2023): \textit{Foundational Work}
\end{itemize}

\vspace{0.5em}
\textbf{Books}
\begin{itemize}
\item Comprehensive Guide
\item Practical Handbook
\item Theory and Practice
\end{itemize}

\column{0.48\textwidth}
\textbf{Online Resources}
\begin{itemize}
\item Official documentation
\item Video tutorials
\item Interactive examples
\item Community forums
\end{itemize}

\vspace{0.5em}
\textbf{Tools}
\begin{itemize}
\item Software package A
\item Library B
\item Framework C
\end{itemize}
\end{columns}

\bottomnote{Curated resources accelerate learning}
\end{frame}

% ==================== SECTION: DATA VISUALIZATION ====================
\section{Data Visualization}

\begin{frame}[t]
\vfill
\centering
\begin{beamercolorbox}[sep=8pt,center]{title}
\usebeamerfont{title}\Large Data Visualization\par
\end{beamercolorbox}
\vfill
\end{frame}

% ==================== LAYOUT 21: FULL-SIZE CHART ====================
\begin{frame}[t]{Full-Size Chart Layout}
\begin{center}
\vspace{0.5em}
% Full-size chart placeholder
\framebox[0.95\textwidth][c]{
\vspace{7cm}
\textcolor{midgray}{[Full-Size Chart/Visualization]}
\vspace{7cm}
}
\end{center}

\bottomnote{Key insight or interpretation of the visualization}
\end{frame}

% ==================== LAYOUT 22: CHART WITH EXPLANATIONS ====================
\begin{frame}[t]{Chart with Bottom Explanations}
\begin{center}
% Main chart placeholder
\framebox[0.95\textwidth][c]{
\vspace{5cm}
\textcolor{midgray}{[Main Chart/Visualization]}
\vspace{5cm}
}
\end{center}

\vspace{0.5em}
\textbf{Key Observations:}
\begin{itemize}
\item Trend 1: Description of first pattern or insight
\item Trend 2: Description of second pattern or insight
\item Trend 3: Description of third pattern or insight
\end{itemize}

\bottomnote{Additional context or methodology notes about the data}
\end{frame}

\end{document}